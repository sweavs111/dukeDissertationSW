\chapter{Basic Document Class Features, Using  a Very Long Chapter Title to Illustrate Appropriate Behaviors}
\label{chap:example}

This chapter is an example of how to format normal material in the
dissertation style.  Most of this information is standard to \LaTeX.

\section{Intra-chapter divisions: Sections}
Section headlines are {\verb|\Large|} and in italics. Compare them
to subsections below. 

\subsection{Subsections: More numbers!}

Isn't it funny that upright letters are
called ``roman'' while slanted letters are ``italic''.  That's like Italian,
and Romans are Italians too.  What gives?  

\subsubsection{Subsubsections: Smaller and smaller}

Subsubsections are allowed, but are not numbered and don't appear in the
table of contents.  Likewise, you can use the next level of sectioning.
\paragraph{Paragraphs} These divisions are unnumbered and do not appear in the Table of Contents. Why would you use this at all?
\subparagraph{Subparagraphs} This is the finest division possible.  It's also
unnumbered and omitted from the Table of Contents. If you find yourself subdividing sections this finely, you have probably made some poor organizational choices. 

\section{Let's do some math}
Let's look at an equation:
\begin{equation}
\label{eq:diffeq}
\partial{f}{t} = f(t) \quad \text{subject to} \quad f(0) = c.
\end{equation}
We've used the {\verb|\newcommand|} defined in the preamble of 
{\verb| dissertation.tex |} to produce the derivative.  You can get a second
derivative like $\partial{^2 f}{t^2}$ by adding some sneaky superscripts.  Fancy.  

More advanced equation formatting is available in the AMS environments.  See
the guide \href{ftp://ftp.ams.org/pub/tex/doc/amsmath/amsldoc.pdf}{amsmath
user's guide}.  Here are some nice examples of cases people usually have
trouble with.

An equation that's too long for one line --- use \texttt{multline}:
\begin{multline}
	a +b+c+d+e+f+g+h+i+j+k+l+m+n+o\\ 
	= p+q+r+s+t+u+v+w+x+y+z.
\end{multline}
An equation with multiple parts and one number per line --- use
\texttt{align}:
\begin{align}
	a_1 &= b_1 + c_1\\
	a_2 &= b_2 + c_2.
\end{align}
The same equation, set inside the \texttt{subequations} environment:
\begin{subequations}
	\label{hello}
	\begin{align}
		\label{goodbye}
		a_1 &= b_1 + c_1\\
		\label{goodbye_b}
		a_2 &= b_2 + c_2.
	\end{align}
\end{subequations}
Notice that by clever placement of labels, I can reference the pair via
\eqref{hello}, the first \eqref{goodbye}, or the second \eqref{goodbye_b}.
One number for multiple equations can be accomplished using the
\texttt{split} environment:
\begin{equation}
	\begin{split}
		a &= b + c - d\\
		 &\phantom{=} + e - f\\
		 &= g + h\\
		 &= i.
	\end{split}
\end{equation}
People often struggle under the complicated and ugly 'eqnarray' environment.
Don't do it!  The AMS ones are easy.  Other stumbling blocks are cases:
\begin{equation}
	a = \begin{cases}
		b & \text{for }x > 0\\
		c & \text{otherwise,}
	\end{cases}
\end{equation}
matrices:
\begin{equation}
	A = \begin{pmatrix} a_{11} & a_{12} \\ a_{21} & a_{22} \end{pmatrix}
     = \begin{bmatrix} a_{11} & a_{12} \\ a_{21} & a_{22} \end{bmatrix},
\end{equation}
and evaluation bars:
\begin{equation}
	a = \frac{\partial u}{\partial x}\bigg\lvert_{x=0}.
	% Note: if you do this a lot, consider defining a command
	% 'eval' to format this the same every time.
\end{equation}
See the source file for details.

When we reference an equation with something like 
{\verb| \eqref |} 
\eqref{eq:diffeq}. If you
click on the above references in the PDF, your viewer should scroll up to
the above equation. It's handy. 
Labels and references may be attached to all
sorts of objects.  There is a {\verb| \label |} attached to this chapter (it
appears at the top of this file), and we may reference it by
({\verb| Chapter~\ref{chap:example} |}), producing
``Chapter~\ref{chap:example}''.  By default these ref's are hyperlinked as
well.  Later, we'll see labeled and referenced figures and tables.
Particular pages may be labeled with standard {\verb| \label |} commands in
the text and referenced via {\verb| \pageref |}.

You might also like the links from 
{\verb| \cite |} commands to the corresponding bibliographic entry.
Go look at this imaginary book by Stephen Colbert \cite{fancy}.  If you're
not a bibtex expert, look in \texttt{mybib.bib} at the {\verb| @ARTICLE |}
that generated this entry.  It shows an example of accents on author names
and how to preserve upper-case for letters in the title.  Other entries show
the use of the {\verb| and |} keyword between author names.  You may order a
particular author's name as either ``first last'' or as ``last, first''.
The actual format of the bibliography is controlled by the 
{\verb| \bibliographystyle{} |} command in \texttt{dissertation.tex}. Here is another reference for \cite{BeckerGrunEtAl03} and another for \cite{BonnEggersEtAl}.


\section{Table of Contents Behavior}
Now is a good time to look back at the Table of Contents.  The Graduate School template does not allow the use of hyperlinks from the Table of Contents throughout the text. However, in the TOC should appear in the 'bookmarks' pane when using a viewer such as Adobe Acrobat. This should be populated with named and numbered sections and subsections
identical to the Table of Contents. 

\section{Figures and footnotes}

Figures are set with very little space between the caption and the bottom of
the included graphic.  This is because most graphics programs pad the edges
of images.  If you find the spacing unsatisfactory, you may always add a bit
manually.  The text of the caption is single-spaced, and the word 'Figure'
is set in small caps.  See Fig.~\ref{fig:example}.  Notice the use of the
nonbreakable space ``{\verb| ~ |}'' between the ``Fig.'' and the reference.
%%%%%%%%%%%%%%%
\begin{figure}[tbp]
\begin{center}
\includegraphics[width=3in]{dukeshield.pdf}
\end{center}
\caption[Short caption for table of figures.]{Longer caption for actual body
of dissertation.  Figure captions should be BELOW the figure.}
\label{fig:example}
\end{figure}
%%%%%%%%%%%%%%

\begin{figure}[tbp]
	\begin{center}
		\includegraphics[width=1.25in]{dukeshield.pdf}
	\end{center}
	\caption{This is a short caption for a small figure.}
	\label{fig:example2}
\end{figure}

Figures (and tables) are examples of 'floats --- objects that \LaTeX\ decides
where to place for you.  You may give \LaTeX\ some hints.  Change the 
{\verb| \begin{figure}[tbp] |} to a {\verb| \begin{figure}[b!] |} to restrict the
placement.  Inside the [ ], you can put the following
\begin{itemize}
\item[t] Allow placement at the top of the page
\item[b] Allow placement at the bottom of the page
\item[h] Allow placement 'here', in the middle of the page close to the text
that the figure environment appears next to.
\item[p] Allow placement on a seperate 'floats page' that has no body text.
\item[!] Tighten the screws on the placement algorithm.  This doesn't force
	things to happen as you say, but it makes it much more likely.  
	Be careful: the bang option can cause figures
   to appear above the chapter title and in other bad locations.
\end{itemize}
Notice that each entry just changes what is \emph{allowed}, 
but no preference among
the entries can be registered.  The default is [tbp], which is a very good
default for a document like this, since floats in the middle of a page trap
too much whitespace for double-spaced text. There is also a prohibition against
having a page with more than 75\% float.  Instead, long floats will
get kicked over onto float pages.  Float pages are often a bad idea, as the
creation of one will often cause a domino effect, with all subsequent
figures appearing on float pages themselves, and all these float pages
appearing together at the end of the chapter.  (This is more like sinking
than floating.)  Avoid this by physically moving where the figure environment
appears in your source file to an earlier location.  Don't be afraid to put the environment before
the first spot you reference it!  Many float problems can be solved by a
combination of relocating the figure environment and a little fiddling with the [ ] options.

Also notice the order of the graphic, caption, and label.  If you deviate
from this, strange things can happen.  The caption of this figure shows the
use of short captions (inside []).  These caption appear in the List of
Tables, while the { } captions appear in the body.  If you omit the [ ]
short caption, the long caption will be used in its place.

Another technical note: since this style sheet is designed for processing
by pdflatex, {\verb| \includegraphics |} looks for \texttt{PDF}s,
\texttt{PNG}s, and \texttt{JPG}s instead of the usual \texttt{PS},
\texttt{EPS}, and \texttt{TIFF} formats.  You can convert existing graphics
with a vareity of tools.  \texttt{PDF} graphics are preferred, as they
scale nicely.  The open-source software Inkscape runs on Mac OSX, Windows,
Linux, and some \textsc{UNIX} variants.  Versions 0.46 and beyond have great
support for creating and editing \texttt{PDF}s.  It can even be used to
convert other docs.

\subsection{List of Figures}
If you've put even one measly figure in your document, grad school rules say
you need a List of Figures.  It's automatically generated for you if you do
a {\verb| \listoffigures |} in the master file (heck, it's there right now).
Go look at the list of figures now.  You should be able to click on the
figure number to warp to the figure.  You'll also see the result of the
'short caption' used above.


\section{Table example}

Just to make sure tables are formatted correctly, 
here's an example of a table float, see Table~\ref{tab:example}.
%%%%%%%%%%%
\begin{table}[t]
\caption[Short table caption appears only in List of Tables.]{Long table
	caption appears on in the body text.  See the short caption in the List
	of Tables.  Table captions need to be ABOVE the table.}
	\label{tab:example}
	\begin{center}
	\begin{tabular}{c|c|c}
			\hline
			Numbers & Letters & Symbols \\ \hline
			1 & a & $\dagger$ \\
			2 & b & $\ddot \smile$ \\
			3 & c & $\times$ \\
			4 & d & $\sharp$
		\end{tabular}
	\end{center}
\end{table}
%%%%%%%%%%%%%
You should note that [b] formatting 
({\verb| \begin{center}[b] |})
can cause floats to appear under the
footnotes.  Try changing it here and see the ugliness.  Tables are identical
to figures, except that the word 'Table' appears in the caption and its
entry is in the List of Tables instead of the List of Figures.

\subsection{Bigger Tables}
You may find that having an extra-wide table is necessary, like \cref{tbl:animals}. You may insert a landscape page using the \verb|\begin{landscape}...\end{landscape}| syntax. This will place items on the page in landscape orientation, but will not rotate the page in the PDF.  Use \verb|\usepackage{lscape}| to keep pages in portrait layout, with text rotated to landscape. IUse\verb|\usepackage{pdflscape}| to rotate these pages within the PDF. When using a landscape environment, do not use the \verb|\textwidth| macro, rather you must use \verb|\linewidth|.

\begin{landscape}
\begin{table}[t]

	\caption[Some random things about random animals]{Some random things about random animals. These animals are random, and so is the other information in this table. Does anyone care? Not really, no.}
	
	\label{tbl:animals}		
	\begin{center}
	\begin{tabular}{L{0.15\linewidth} L{0.24\linewidth} L{0.15\linewidth} L{0.15\linewidth} L{0.2\linewidth}}
			\toprule
			\textbf{Name} & \textbf{Animal Name}& \textbf{Scientific Name }& \textbf{Date} & \textbf{Random} \\
			\hline
			Alix & White spoonbill & \textit{Platalea leucordia} & 5/15/2023 & knowledge base \\
			\hline
			Kerk & Kangaroo, eastern grey & \textit{Macropus giganteus} & 6/28/2023 & installation \\
			\hline
			Sol & Puffin, horned & \textit{Fratercula corniculata} & 1/9/2023 & conglomeration \\
			\hline
			Engracia & Ornate rock dragon & \textit{Ctenophorus ornatus} & 9/12/2023 & encompassing \\
			\hline
			Alfred & Vicuna & \textit{Vicugna vicugna} & 5/13/2023 & local area network \\
			\hline
			Kelsey & Tyrant flycatcher & \textit{Myiarchus tuberculifer} & 5/9/2023 & process improvement \\
			\hline
			Dwayne & Stork, saddle-billed & \textit{Ephipplorhynchus senegalensis} & 9/11/2023 & maximized \\
			\hline
			Nolie & Polecat, african & \textit{Ictonyx striatus} & 11/27/2022 & system engine \\
			\hline
			Nicko & Black-backed jackal & \textit{Canis mesomelas }& 3/31/2023 & hardware \\
			\hline
			Marlowe & Trotter, lily & \textit{Actophilornis africanus} & 6/8/2023 & 6th generation \\
			\hline
			Enrica & Bettong, brush-tailed & \textit{Bettongia penicillata }& 8/25/2023 & Secured \\
			\hline
			Theresina & Fox, grey & \textit{Vulpes cinereoargenteus }& 11/20/2022 & Fundamental \\
			\hline
			Syd & Boa, cook's tree & \textit{Corallus hortulanus cooki }& 3/22/2023 & service-desk \\
			\hline
			Ekaterina & Asian elephant & \textit{Elephas maximus bengalensis} & 6/28/2023 & Econometrics \\
			\hline
			Torie & Emerald green tree boa & \textit{Boa caninus} & 10/16/2022 & Databases \\
						\bottomrule
		\end{tabular}
\end{center}
\end{table}
\end{landscape}

\subsection{Footnotes}
Footnotes are allowed.\footnote{But, you should probably just work them into
the text since it's annoying to jump around when reading.} They are
numbered with arabic numerals inside each chapter and appear at the bottom
of the page.\footnote{\ldots rather than the end of the chapter or the
thesis.  Those would properly be endnotes, I guess.}  The little footnote
numbers are also hyperlinks.  Try clicking them.  You should place the
footnote command immediately following the period of the sentence it is
attached to.  Any spaces or newlines will result in strange spacing between
the number and the sentence.

\section{Corner cases in formatting, such as very very very long section titles.  Man, this goes on forever.}

Common corner-cases involve very long titles (like above).  In these cases,
the long titles are set single-spaced both here and in the Table of
Contents.

\subsection{Figure and Table caption cases are neat, and this is an absurdly
long subsection heading}
\label{sec:fig-tab}

Consider the shield logo again with an absurd caption, as in
Fig.~\ref{fig:shield}.
Also examine the new table, Table~\ref{tab:long-caption}.  Both of these
have been forced onto a floats page so you can see what that looks like.
\begin{figure}[p]
	\begin{center}
	\includegraphics[height=1.5in]{dukeshield.pdf}
	\end{center}
	\caption[Duke Logo]{The Duke logo again, but now with a really long rambling
	caption.  This caption should be set single-spaced in the LoF, maximum one line, and in the
	body text.  What do you think about having graphics in the main directory
	of a project?  I'd prefer them in a folder, then put
	'foldername/picturename' as the argument to includegraphics.}
	\label{fig:shield}
\end{figure}
%%%%%%%%%%%
\begin{table}[p]
\caption[This is a silly table, but note that it needs a shorter caption! Maximum 2 lines in the List of Tables]{The same silly table again, but with a really really really
	really really really
	really really really
	really really really
	really really really
	really really really
	really really really
	long caption.}
	\label{tab:long-caption}
	\begin{center}
		\begin{tabular}{c|c|c}
			\hline
			Numbers & Letters & Symbols \\ \hline
			1 & a & $\dagger$ \\
			2 & b & $\ddot \smile$ \\
			3 & c & $\times$ \\
			4 & d & $\sharp$ \\
		\end{tabular}
	\end{center}
\end{table}
%%%%%%%%%%%%%

